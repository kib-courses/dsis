% ВАЖНО
% Не меняйте ничего в этом файле. А если меняете, то делайте это в этом проекте:
% https://github.com/kib-courses/latex_templates
% Для пользовательских настроек есть файл ./header/user.tex
\documentclass{beamer}
\usetheme{metropolis} 
\usecolortheme{rose}

\hypersetup{unicode=true}
\usepackage{tikz}

\usepackage{xcolor}
\usepackage[utf8]{inputenc}
\usepackage{hyphenat}
\usepackage[russian,english]{babel}          % Use metropolis theme
\usepackage{wrapfig}

\usepackage[normalem]{ulem}  % для зачекивания текста

\usepackage{caption}
\captionsetup[figure]{name=Рисунок }
\newcommand{\рис}[1]{рис.\ref{#1}}
\newcommand{\Рис}[1]{Рис.\ref{#1}}


\captionsetup[table]{name=Таблица~№}
\newcommand{\таблицa}[1]{таблица~№\ref{#1}} % именительный падеж
\newcommand{\таблицы}[1]{таблицы~№\ref{#1}} % родительный падеж
\newcommand{\таблице}[1]{таблице~№\ref{#1}} % дательный и предложный падеж
\newcommand{\таблицу}[1]{таблицу~№\ref{#1}} % винительный падеж
\newcommand{\таблицей}[1]{таблицей~№\ref{#1}} % творительный падеж 
\newcommand{\Таблицa}[1]{Таблица~№\ref{#1}} % именительный падеж
\newcommand{\Таблицы}[1]{Таблицы~№\ref{#1}} % родительный падеж
\newcommand{\Таблице}[1]{Таблице~№\ref{#1}} % дательный и предложный падеж
\newcommand{\Таблицу}[1]{Таблицу~№\ref{#1}} % винительный падеж
\newcommand{\Таблицей}[1]{Таблицей~№\ref{#1}} % творительный падеж 

\setbeamertemplate{footline}[frame number] % указывает на каждой странице общее количество страниц

% Указывайте все новые термины в \termdef команде. А уже известные ранее или из других курсов в \term
\newcommand{\termdef}[1]{\textbf{\textit{#1}}}
\newcommand{\term}{\textit}

% Диалог с аудиторией.
\newcommand{\auditorium}[1]{\color{red}{\textbf{#1}}}

\let\OLDhref\href
\renewcommand{\href}[2]{\textcolor{blue}{\OLDhref{#1}{#2}}}

% \setbeameroption{show notes}
% \usepackage{listings}             % Include the listings-package
% \usepackage{minted}

\usepackage{CJKutf8}

\title{Лекция 3. RSA Risk Engine, RSA Rules Engine}
% \date{\today}
\date{8 октября 2019}
\author{Павел Владимирович Слипенчук}
\institute{Москва, МГТУ им.Бауманка,\\ каф.ИУ-8, КИБ}
% \titlegraphic{\includegraphics[width=2cm]{logo_ur.jpg}}
\titlegraphic{\small \href{https://github.com/kib-courses/dsis}{Data Science для решения задач информационной безопасности}}

\begin{document}
  \maketitle
    
\begin{frame}{План лекции}
    \begin{enumerate}
		\item \nameref{section:feedback}
	% \item \nameref{section:precision_recall_defs}
	% \item \nameref{section:tree_forest}
	% \item \nameref{section:random_tree_building}
	% \item \nameref{section:random_forest}
	\end{enumerate}
\end{frame}

\section{Обратная связь}\label{section:feedback}

\begin{frame}{Историческая заметка}
	\begin{columns}
		\begin{column}{0.5\textwidth}
			\includegraphics[width=5cm]{../pic/anohin.jpg}
		\end{column}
		\begin{column}{0.6\textwidth}
			\begin{enumerate}
				\item В 1935 году советский физиолог
				Пётр Кузьмич Анохин формулирует понятие обратной связи.
				\item Термин «перекочевал» в Кибернетику (1948) Н.Винера
				\item В настоящее время термин «обратная связь» используется в медицине, в технике, в акустике, в биологии, в социальных науках.
			\end{enumerate}
		\end{column}
	\end{columns}
\end{frame}

\begin{frame}{Обратная связь в DS}
	\termdef{Обратная связь (feedback) I}  --
	это любая информация, используемая для оценки качества ЭС
	и/или для получения новой \term{обучающей выборки} (или выборки \term{обучения с подкреплением})
	
	\termdef{Обратная связь (feedback) II} -- это процесс, система, 
	программный модуль, реализующий обратную связь I
\end{frame}

\begin{frame}{Обучение с подкреплением, reinforcement learning (Дообучение, Refitting)\footnote{слайд из лекции №2}}
	\small
	Есть первоначальная обучающая выборка: 
	\begin{equation*}
		U_{fit} = \left\{ y \mapsto \bold x  \right\}
	\end{equation*}
	Первоначальное обучение:
	\begin{equation*}
		score_1 := fit (U_{fit}) 
	\end{equation*} 
	В дальнейшем при получении нового множества $\hat U_{fit}$ (возможно состоящее из одного элемента)
	мы дообучаем систему:
	\begin{equation}
		score_{i+1} := refit (\hat U_{fit}, score_{i}) 
	\end{equation} 
	Таким образом функция $score_i$ заменяется на новую функцию $score_{i+1}$.
\end{frame}

\begin{frame}{Итеративное обучение}
	\small
	Есть первоначальная обучающая выборка: 
	\begin{equation*}
	\hat U_1 = U_{fit} = \left\{ y \mapsto \bold x  \right\}
	\end{equation*}
	Первоначальное обучение:
	\begin{equation*}
	score_1 :=   fit (\hat U_{1}) = fit (U_{fit}) 
	\end{equation*} 
	В дальнейшем при получении нового множества $\hat U_{i}$ (возможно состоящее из одного элемента)
	мы повторно обучаем систему:
	\begin{equation}
	score_{i+1} := fit ( U_{fit} \cup \hat U_2 \cup ... \cup  \hat U_{i}) 
	\end{equation} 
	Таким образом функция $score_i$ заменяется на новую функцию $score_{i+1}$.
\end{frame}

\begin{frame}
	\auditorium{Верно ли утверждение что итеративное обучение позволяет для любого алгоритма
	осуществить обучение с подкреплением?}
\end{frame}

\begin{frame}{Стабильность. <<Цепочка>>}\label{frame:chain_score}
	Есть итеративная система обучения и имеем скоринги:
	\begin{equation*}
	score_{i-n} := fit ( \hat U_{1} \cup \hat U_2 \cup ... \cup  \hat U_{i-n} \cup ... \cup \hat U_{i-1} \cup \hat U_{i}) 
	\end{equation*}
	\begin{equation*}
	...
	\end{equation*}
	\begin{equation*}
	score_{i-1} := fit ( \hat U_{1} \cup \hat U_2 \cup ... \cup  \hat U_{i-1} \cup \hat U_{i}) 
	\end{equation*}
	\begin{equation*}
	score_{i} := fit ( \hat U_{1} \cup \hat U_2 \cup ... \cup  \hat U_{i}) 
	\end{equation*}
	
	Тогда задав веса $w_j > 0$ можно сделать простой ансамбль вида:
	\begin{equation*}
	\bold {score_{i}}(\bold x) = \frac{w_1 \cdot score_{i-n}(\bold x) + ... + w_n \cdot score_{i}(\bold x)}{\sum_{j=1}^{n} w_j} 
	\end{equation*}
	
	Например:
	\begin{equation*}
	\bold {score_{i}}(\bold x) = 
		0.2 \cdot score_{i-2}(\bold x) 
		+ 0.3 \cdot score_{i-1}(\bold x) 
		+ 0.5 \cdot score_{i}(\bold x) 
	\end{equation*}
\end{frame}


\begin{frame}
	Возвращаемся к обратной связи
	\Large
	\begin{center}
		\auditorium{Зачем нужна обратная связь в DS?}
	\end{center}
\end{frame}

\begin{frame}
	Feedback позволяет:
	\begin{enumerate}
		\item Изменить $score$ на основании предыдущих \term{откликов}
		и размеченных данных. Например при каждой сработке 
		можно с вероятностью $p$ проводить расследование и дообучать систему.
		\item генетические алгоритмы: изменение параметров
		\item позволяет выявлять аномалии\footnote{<<аномалитика>> :) },
		корректировать мат.модель, выбирать из ML моделей лучшую.
	\end{enumerate}
\end{frame}


\section{RSA Risk Engne}

\begin{frame}{Общая схема} 
	\includegraphics[width=14cm]{../pic/expert_system_rsa.png}
	% \includegraphics[width=5cm]{../pic/anohin.jpg}	
\end{frame}

\begin{frame}{Последовательность}
\end{frame}

\section{RSA Rules Engine}


  
\section{Вопросы для самопроверки}

\begin{frame}
	Прочитайте, что такое <<генетические алгоритмы>>(genetic algorithm).
	В чём роль обратной связи в них?
	
	Предложите структуру гена генетического алгоритма, создающий стратегии игры в крестики нолики. 
	
	Для желающих: реализуйте его (вместо какой-нибудь лабораторной). 
	<<стравите>> разные гены друг с другом и путём <<эволюции>>
	получите самый сильный алгоритм.
	
	Стал ли полученный алгоритм оптимальным? Сыграйте несколько партий. Сколько раз выиграли, 
	сколько проиграли, сколько раз сыграли вничью?
	
\end{frame}

\begin{frame}{Обратная связь и итеративное обучение}
	\begin{enumerate}
		\item В <<цепочке>> (см. слайд №\ref{frame:chain_score}) задана функция
		$\bold score$ с коэффициентами $(0.1, 0.2, 0.7)$. Каков будет скоринг, 
		если $score_1=540$, $score_2=546$, $dscore_3=584$?
		\item В <<цепочке>> задана функция 	$\bold score$ 
		с коэффициентами $(0.2, 0.4, 0.8)$. Значения $score_1, score_2, score_3$
		равны $345$, $124$ и $573$. Почему $\bold score$ 
		будет $412$, 
		а не $577$ ?
		\item Предположим, что $score_{i+1}$ 
		сильно больше чем $score_i$. Что это значит? 
		Действительно ли система работает нестабильно? 
		Приведите контрпримеры.
		
	\end{enumerate}
\end{frame}

\end{document}