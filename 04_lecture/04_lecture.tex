% ВАЖНО
% Не меняйте ничего в этом файле. А если меняете, то делайте это в этом проекте:
% https://github.com/kib-courses/latex_templates
% Для пользовательских настроек есть файл ./header/user.tex
\documentclass{beamer}
\usetheme{metropolis} 
\usecolortheme{rose}

\hypersetup{unicode=true}
\usepackage{tikz}

\usepackage{xcolor}
\usepackage[utf8]{inputenc}
\usepackage{hyphenat}
\usepackage[russian,english]{babel}          % Use metropolis theme
\usepackage{wrapfig}

\usepackage[normalem]{ulem}  % для зачекивания текста

\usepackage{caption}
\captionsetup[figure]{name=Рисунок }
\newcommand{\рис}[1]{рис.\ref{#1}}
\newcommand{\Рис}[1]{Рис.\ref{#1}}


\captionsetup[table]{name=Таблица~№}
\newcommand{\таблицa}[1]{таблица~№\ref{#1}} % именительный падеж
\newcommand{\таблицы}[1]{таблицы~№\ref{#1}} % родительный падеж
\newcommand{\таблице}[1]{таблице~№\ref{#1}} % дательный и предложный падеж
\newcommand{\таблицу}[1]{таблицу~№\ref{#1}} % винительный падеж
\newcommand{\таблицей}[1]{таблицей~№\ref{#1}} % творительный падеж 
\newcommand{\Таблицa}[1]{Таблица~№\ref{#1}} % именительный падеж
\newcommand{\Таблицы}[1]{Таблицы~№\ref{#1}} % родительный падеж
\newcommand{\Таблице}[1]{Таблице~№\ref{#1}} % дательный и предложный падеж
\newcommand{\Таблицу}[1]{Таблицу~№\ref{#1}} % винительный падеж
\newcommand{\Таблицей}[1]{Таблицей~№\ref{#1}} % творительный падеж 

\setbeamertemplate{footline}[frame number] % указывает на каждой странице общее количество страниц

% Указывайте все новые термины в \termdef команде. А уже известные ранее или из других курсов в \term
\newcommand{\termdef}[1]{\textbf{\textit{#1}}}
\newcommand{\term}{\textit}

% Диалог с аудиторией.
\newcommand{\auditorium}[1]{\color{red}{\textbf{#1}}}

\let\OLDhref\href
\renewcommand{\href}[2]{\textcolor{blue}{\OLDhref{#1}{#2}}}

% \setbeameroption{show notes}
% \usepackage{listings}             % Include the listings-package
% \usepackage{minted}

\usepackage{CJKutf8}

\title{Лекция 4. UEBA: анализ непрерывных потоков (track analysis). Mouse Track Analysis. Mobile Entity Analitics}

% \date{\today}
\date{22 октября 2019}
\author{Павел Владимирович Слипенчук \\ Ольга Сергеевна Шкряба}
\institute{Москва, МГТУ им.Бауманка,\\ каф.ИУ-8, \href{https://t.me/kibinfo}{КИБ}}
% \titlegraphic{\includegraphics[width=2cm]{logo_ur.jpg}}
\titlegraphic{\small \href{https://github.com/kib-courses/dsis}{Data Science для решения задач информационной безопасности}}

\begin{document}
  \maketitle
    
\begin{frame}{План лекции}
    \begin{enumerate}
    	\item \nameref{section:ueba}
    	\item \nameref{section:mca}
		\item \nameref{section:mea}
		\item \nameref{section:ueba_other}
	\end{enumerate}
\end{frame}

\section{UEBA}\label{section:ueba}
	\begin{center}
		\includegraphics[width=7cm]{../pic/Jackie_Chan.png}
		
		\LARGE
		\auditorium{UEBA?}
	\end{center}

\begin{frame}{User and Entity Behavioral Analitics}

	\termdef{Потоковые данные} (stream data) -- 
	<<локально связные данные>> пользовательской активности
	(трек мыши, последовательность действий, keystroke dynamics и т.д.)
	
	\begin{equation*}
	track~data \subseteq stream~data \subseteq raw~data
	\end{equation*}
	
	\termdef{Поведенческая аналитика}, User and Entity Behavioral Analitics (UEBA) --
	процесс сбора, ручного анализа и построения экспертных систем
	основанных на больших объёмов \term{потоковых данных}, генерируемые пользовательскими приложениями
\end{frame}

\begin{frame}{Примеры UEBA в ИБ}
	\begin{enumerate}
		\item MTA (Mouse Track Analysis) --
		анализ трека мыши.
		Например на Web-портале (в частности ДБО).
		\item keystroke dynamics --
		анализ клавиатурного почерка.
		Например при вводе логина и пароля в ДБО.
		\item predilections (предпочтения) --
		предпочтения пользователя. 
		Например как копирует информацию: ctrl+v, shift+insert или кнопка мыши + "вставить"?
	\end{enumerate}
\end{frame}

\begin{frame}{Определение точности для задачи идентификации}
	\small
	Пусть имеется множество классов (не обязательно два): $1, 2, 3, ..., n$.
	
	Введём события:
	\begin{itemize}
		\item $L_r^i$ -- событие, что объект действительно пренадлежит классу $i$;
		\item $L_s^i$ -- событие, что экспертная система определила событие к классу $i$.
	\end{itemize}
	
	Тогда \termdef{точность обнаружения своих для класса $i$} ($T$)
	% и \termdef{полнота обнаружения для класса $i$} ($\Pi$)
	определяется по формуле:
	\begin{equation}
	\hat T_i \stackrel{def}{=} P (L_r^i | L_s^i)
	\end{equation}
	% \begin{equation}
	% \Pi_i \stackrel{def}{=} P (L_s^i | L_r^i)
	% \end{equation}
\end{frame}

\begin{frame}
	\small
	Определим операцию $[cond]$: 
	\begin{itemize}
		\item $[cond]=1$ если $cond$ истинно 
		\item $[cond]=0$, если $cond$ ложно
	\end{itemize}
	
	определим величину $refusal \in [0, 1]$.
	
	Будем считать, что $n >> 0$.
	
	Тогда \termdef{точность обнаружения своих} при отказе $refusal$ это такая максимально
	возможная величина $\hat T$, что выполняется уравнение:
	\begin{equation}
	\frac{\sum_{i=1}^{n}[\hat T_i > 
		\hat T]}{n} \geqslant 1 -  refusal
	\end{equation}
\end{frame}

\begin{frame}
	\Large
	На практике интересны системы. для которых 
	\term{точность для своих} $ \approx 99.9...9\%$.
	
	\begin{center}
		\auditorium{Почему?}
	\end{center}
	Таким образом важным параметром является величина отказа
	(величина $refusal$)
	
	Таким образом разумно зафиксировать 
	$ \hat T = 99.9...9\%$ и уже \textbf{после}
	подбирать $refusal$.
	
	\auditorium{Объясните физический смысл значения refusal в этом случае.}
\end{frame}

\begin{frame}{Точность [определения чужих]}
	\small
	Предположим, что в некоторой системе под учётной записью 
	работает \termdef{другой} человек.
	Не факт, что это фрод (событие F -- fraud).
	
	Обозначим его через букву $A$ (от англ \textbf{another} или \textbf{anomaly})
	
	Введём ещё два события:
	\begin{itemize}
		\item $A_r^i$ (another) -- событие, что в действительности под учётной записью $i$
		работает пользователь $\neq i$;
		\item $A_s^i$ (anomaly)-- событие, что экспертная система определила событие
		как аномальное для пользователя $i$
	\end{itemize}

	Тогда \termdef{точность [для определения чужих]}	
	для пользователя $i$	

	\begin{equation}
	\hat T_i \stackrel{def}{=} P (A_r^i | A_s^i)
	\end{equation}
	
\end{frame}


\begin{frame}{Фундаментальная проблема всех UEBA решений}
	\begin{block}{Очень важное замечание.}
		Обычно \href{https\%3A\%2F\%2Fru.wikipedia.org\%2Fwiki\%2F\%D0\%9F\%D1\%80\%D0\%BE\%D0\%BA\%D0\%BB\%D1\%8F\%D1\%82\%D0\%B8\%D0\%B5_\%D1\%80\%D0\%B0\%D0\%B7\%D0\%BC\%D0\%B5\%D1\%80\%D0\%BD\%D0\%BE\%D1\%81\%D1\%82\%D0\%B8}{Проклятье Размерности (ПР)}
		рассматривают проблему количества признаков в Data Science.
		
		Однако частный случай ПР -- это большое количество \term{классов}.
		Многие задачи тривиально решаются при малом количестве классов, 
		но полнота и точность падают к 0\% при существенном увеличении количества классов.
		
		Не обольщайтесь на академические работы!
	\end{block}


\end{frame}


\section{Mouse Track Analysis}\label{section:mca}


\section{Mobile Entity Analitics}\label{section:mea}
% TODO часть Ольги Шкряба

\section{Другие UEBA задачи анализа непрерывных потоков}\label{section:ueba_other}
% Eye Tracking
% Полиграф
% ...

\section{Вопросы для самопроверки}

\end{document}