% ВАЖНО
% Не меняйте ничего в этом файле. А если меняете, то делайте это в этом проекте:
% https://github.com/kib-courses/latex_templates
% Для пользовательских настроек есть файл ./header/user.tex
\documentclass{beamer}
\usetheme{metropolis} 
\usecolortheme{rose}

\hypersetup{unicode=true}
\usepackage{tikz}

\usepackage{xcolor}
\usepackage[utf8]{inputenc}
\usepackage{hyphenat}
\usepackage[russian,english]{babel}          % Use metropolis theme
\usepackage{wrapfig}

\usepackage[normalem]{ulem}  % для зачекивания текста

\usepackage{caption}
\captionsetup[figure]{name=Рисунок }
\newcommand{\рис}[1]{рис.\ref{#1}}
\newcommand{\Рис}[1]{Рис.\ref{#1}}


\captionsetup[table]{name=Таблица~№}
\newcommand{\таблицa}[1]{таблица~№\ref{#1}} % именительный падеж
\newcommand{\таблицы}[1]{таблицы~№\ref{#1}} % родительный падеж
\newcommand{\таблице}[1]{таблице~№\ref{#1}} % дательный и предложный падеж
\newcommand{\таблицу}[1]{таблицу~№\ref{#1}} % винительный падеж
\newcommand{\таблицей}[1]{таблицей~№\ref{#1}} % творительный падеж 
\newcommand{\Таблицa}[1]{Таблица~№\ref{#1}} % именительный падеж
\newcommand{\Таблицы}[1]{Таблицы~№\ref{#1}} % родительный падеж
\newcommand{\Таблице}[1]{Таблице~№\ref{#1}} % дательный и предложный падеж
\newcommand{\Таблицу}[1]{Таблицу~№\ref{#1}} % винительный падеж
\newcommand{\Таблицей}[1]{Таблицей~№\ref{#1}} % творительный падеж 

\setbeamertemplate{footline}[frame number] % указывает на каждой странице общее количество страниц

% Указывайте все новые термины в \termdef команде. А уже известные ранее или из других курсов в \term
\newcommand{\termdef}[1]{\textbf{\textit{#1}}}
\newcommand{\term}{\textit}

% Диалог с аудиторией.
\newcommand{\auditorium}[1]{\color{red}{\textbf{#1}}}

\let\OLDhref\href
\renewcommand{\href}[2]{\textcolor{blue}{\OLDhref{#1}{#2}}}

% \setbeameroption{show notes}
% \usepackage{listings}             % Include the listings-package
% \usepackage{minted}

\usepackage{CJKutf8}

\title{Лекция 4. UEBA: анализ непрерывных потоков (track analysis). Mouse Track Analysis. Mobile Entity Analitics}

% \date{\today}
\date{22 октября 2019}
% \author{Павел Владимирович Слипенчук \\ Ольга Сергеевна Шкряба}
\author{Павел Владимирович Слипенчук}

\institute{Москва, МГТУ им.Бауманка,\\ каф.ИУ-8, \href{https://t.me/kibinfo}{КИБ}}
% \titlegraphic{\includegraphics[width=2cm]{logo_ur.jpg}}
\titlegraphic{\small \href{https://github.com/kib-courses/dsis}{Data Science для решения задач информационной безопасности}}

\begin{document}
  \maketitle
    
\begin{frame}{План лекции}
    \begin{enumerate}
    	\item \nameref{section:ueba}
    	\item \nameref{section:mca}
		\item \nameref{section:mea}
		% \item \nameref{section:ueba_other}
	\end{enumerate}
\end{frame}

\section{UEBA}\label{section:ueba}
	\begin{center}
		\includegraphics[width=7cm]{../pic/Jackie_Chan.png}
		
		\LARGE
		\auditorium{UEBA?}
	\end{center}

\begin{frame}{User and Entity Behavioral Analitics}

	\termdef{Потоковые данные} (stream data) -- 
	<<локально связные данные>> пользовательской активности
	(трек мыши, последовательность действий, keystroke dynamics и т.д.)
	
	\begin{equation*}
	track~data \subseteq stream~data \subseteq raw~data
	\end{equation*}
	
	\termdef{Поведенческая аналитика}, User and Entity Behavioral Analitics (UEBA) --
	процесс сбора, ручного анализа и построения экспертных систем
	основанных на больших объёмах \term{потоковых данных}, генерируемые пользовательскими приложениями
\end{frame}

\begin{frame}{Примеры UEBA в ИБ}
	\begin{enumerate}
		\item MTA (Mouse Track Analysis) --
		анализ трека мыши.
		Например на Web-портале (в частности ДБО).
		\item keystroke dynamics --
		анализ клавиатурного почерка.
		Например при вводе логина и пароля в ДБО.
		\item predilections (предпочтения) --
		предпочтения пользователя. 
		Например как копирует информацию: ctrl+v, shift+insert или кнопка мыши + "вставить"?
	\end{enumerate}
\end{frame}

\begin{frame}{Определение точности для задачи идентификации}
	\small
	Пусть имеется множество классов (не обязательно два): $1, 2, 3, ..., n$.
	
	Введём события:
	\begin{itemize}
		\item $L_r^i$ -- событие, что объект действительно пренадлежит классу $i$;
		\item $L_s^i$ -- событие, что экспертная система определила событие к классу $i$.
	\end{itemize}
	
	Тогда \termdef{точность обнаружения своих для класса $i$} ($\hat T$)
	% и \termdef{полнота обнаружения для класса $i$} ($\Pi$)
	определяется по формуле:
	\begin{equation}\label{eq:ueba_self_presicion_def}
	\hat T_i \stackrel{def}{=} P (L_r^i | L_s^i)
	\end{equation}
	% \begin{equation}
	% \Pi_i \stackrel{def}{=} P (L_s^i | L_r^i)
	% \end{equation}
\end{frame}

\begin{frame}
	\small
	Определим операцию $[cond]$: 
	\begin{itemize}
		\item $[cond]=1$ если $cond$ истинно 
		\item $[cond]=0$, если $cond$ ложно
	\end{itemize}
	
	определим величину $refusal \in [0, 1]$.
	
	Будем считать, что $n >> 0$.
	
	Тогда \termdef{точность обнаружения своих} при отказе $refusal$ это такая максимально
	возможная величина $\hat T$, что выполняется уравнение:
	\begin{equation}\label{eq:ueba_self_precision_equation}
	\frac{\sum_{i=1}^{n} \left[\hat T_i > 
		\hat T\right]}{n} \geqslant 1 -  refusal
	\end{equation}
	Так как эта величина \term{максимально возможная}, то
	разумно формулу \eqref{eq:ueba_self_precision_equation} 
	переписать так:
	\begin{equation}
	\frac{\sum_{i=1}^{n} \left[\hat T_i > 
		\hat T\right]}{n} = 1 -  refusal
	\end{equation}
	
\end{frame}

\begin{frame}
	\Large
	На практике интересны системы. для которых 
	\term{точность для своих} $ \approx 99.9...9\%$.
	
	\begin{center}
		\auditorium{Почему?}
	\end{center}
	Таким образом важным параметром является величина отказа
	(величина $refusal$)
	
	Таким образом разумно зафиксировать 
	$ \hat T = 99.9...9\%$ и уже \textbf{после}
	подбирать $refusal$.
	
	\auditorium{Объясните физический смысл значения refusal в этом случае.}
\end{frame}

\begin{frame}{Точность [определения чужих]}
	\small
	Предположим, что в некоторой системе под учётной записью 
	работает \termdef{другой} человек.
	Не факт, что это фрод (событие F -- fraud).
	
	Обозначим его через букву $A$ (от англ \textbf{another} или \textbf{anomaly})
	
	Введём ещё два события:
	\begin{itemize}
		\item $A_r^i$ (another) -- событие, что в действительности под учётной записью $i$
		работает пользователь $\neq i$;
		\item $A_s^i$ (anomaly)-- событие, что экспертная система определила событие
		как аномальное для пользователя $i$
	\end{itemize}

	Тогда \termdef{точность [определения чужих]}	
	для пользователя $i$	

	\begin{equation}\label{eq:ueba_presicion_def}
	T_i \stackrel{def}{=} P (A_r^i | A_s^i)
	\end{equation}
\end{frame}

\begin{frame}{Точность [определения чужих]}
	\small
	Возьмём все классы, которые не попадают в $refusal$.
	
	Определим величину $n_1$:
	\begin{equation}
	n_1 \stackrel{def}{=} \sum_{i=1}^{i=n} \left[\hat T_i > \hat T \right]	
	\end{equation}
	
	\auditorium{ДЗ. Докажите, что верно уравнение}:
	\begin{equation}
	refusal = 1 - \frac{n_1}{n}
	\end{equation}
	
	Точностью [определения чужих] считается средняя величина:
	\begin{equation}\label{eq:ueba_presicion_def_equation}
	T = \frac{\sum_{i=1}^{n} T_i \cdot \left [\hat{T_i} > \hat T \right]}{n_1} 
	\end{equation}
	
	Или медиана.
	\auditorium{ДЗ. Самостоятельно напишите формулу, аналогичную \eqref{eq:ueba_presicion_def_equation}}
\end{frame}

\begin{frame}{Примеры ошибок}
	\small
	\begin{enumerate}
		\item очень часто просто не декларируется величина отказ ($refusal$).
		Это абсурд. Всегда есть странные сессии и даже странные пользователи,
		для которых  UEBA не применимо в принципе
		\item путают точность для своих и точность [для чужих]
		\item не понимают, что почти всегда $\hat T$ должна равняетcя почти $1$.
		И уже из этого соображения рассчитывать $refusal$.
		\item точность [для чужих] можно рассчитывать как мат.ожидание, так и как медиану.
		Или другим способом.
		Иногда существует большой класс данных для которых точность невероятно мала 
		(например 2/3), но для другой части (например 1/3) точность очень высока (например $T=90\%$)
		Очевидно что такая система имеет практическую пользу 
		(при не очень большом $refusal$ разумеется), однако медиана будет очень низкой,
		а средняя величина всего чуть больше $30\%$.
	\end{enumerate}
\end{frame}

\begin{frame}{Фундаментальная проблема всех UEBA решений}
\begin{block}{<<Очень важное замечание.>> (О.В.З)}
	Обычно \href{https\%3A\%2F\%2Fru.wikipedia.org\%2Fwiki\%2F\%D0\%9F\%D1\%80\%D0\%BE\%D0\%BA\%D0\%BB\%D1\%8F\%D1\%82\%D0\%B8\%D0\%B5_\%D1\%80\%D0\%B0\%D0\%B7\%D0\%BC\%D0\%B5\%D1\%80\%D0\%BD\%D0\%BE\%D1\%81\%D1\%82\%D0\%B8}{Проклятье Размерности (ПР)}
	рассматривают как проблему количества признаков в Data Science.
	
	Однако частный случай ПР -- это большое количество \term{классов}.
	Многие задачи тривиально решаются при малом количестве классов, 
	но полнота и точность падают к 0\% при существенном увеличении количества классов.
	
	Не обольщайтесь на академические работы!
\end{block}
\end{frame}

\begin{frame}{Сказка о глупом мышонке\footnote{основана на реальных событиях. Нет, это произошло не со мной :)}}
	\footnotesize
	Жила-была Мышь-датасаентолог. 
	
	Решила она создать UEBA систему распознавания почерка движения мыши(компьютерной) 
	для мышей(настоящих).  	
	Мышь собрала \textit{нескольких} своих друзей-мышек, 
	которые по 30 минут играли в игру, используя мышь(компьютерную).
	$refusal$ у системы $0\%$ (мышь-датасаентолог и не думала рассчитывать эту величину),
	$\hat T = 100\%$, а вот средняя $T = 76\%$.
	
	Мышь идёт к Лосю-инвестору. Лось-инвестор в восторге
	и на дофига тугриков нанимает полк мышей для внедрения 
	данной UEBA системы
	в <<Процессинг X>>,
	в котором обслуживаются 5 млн. тушканчиков.
	
	В конце epic fail: система не заработала. 
	Лось потерял все свои деньги. Мышь-датасантист насильно отправлена в мышеловку.
	
	\auditorium{В чём мораль сказки? В чём ошиблась мышь? И какие вопросы должен был задать Лось?}
\end{frame}

\begin{frame}{Маркетинг и реальность}
	К сожалению UEBA -- это хайп. 
	Им занимаются все, кому не лень. 
	
	Доходит до смешного: просто говорят явные глупости.
	Например мы распознаем ВСЕХ пользователей.
	
	Очень часто об UEBA говорят просто потому что это модно.
	
	\auditorium{Здравый подход к UEBA. Каков он?}
\end{frame}

\begin{frame}{Д.Н.К.}
	Любая DS задача, в частности по UEBA данным, ставится в парадигме Д.Н.К.:
	\begin{itemize}
		\item \textbf{Дано}. Какие есть данные?
		Точно только эти? Может есть ещё какие-нибудь данные?
		Насколько легко создать процессинг по <<переламыванию>>
		этих данных?
		\item \textbf{Найти}. Что именно требуется найти?
		Почему именно это нужно найти? Может что-то другое?
		\item \textbf{Критерий}. Каков критерий того, что мы нашли действительно
		то, что искали? Насколько этот критерий объективен? Можно ли автоматизировать
		процесс расчёта этого критерия? Если нет, то как часто нужно <<блюсти>>
		систему?
	\end{itemize}
	На счёт UEBA добавляется вопрос: а зачем? Без UEBA можно обойтись?
\end{frame}

\begin{frame}{Последовательность решения UEBA задачи}
	\begin{enumerate}
		\item определяется Д.Н.К.
		\item выгружаются данные. Проводится Feature Extraction. 
		Любыми способами пытаемся решить задачу для $n=5$.
		\item смотрим тенденцию роста $refusal$ и падения $T$
		при увеличении $n$: $n=5$, $n=50$, $n=250$, ...
		\item находится решение <<стабилизации>>
		 $refusal$ и  $T$ для $n \longrightarrow \infty$
		\item тестирование на очень большой ($n >> 0$) выборке 
		\item внедрение
	\end{enumerate}
	\begin{block}{Замечание.}
		Ещё раз, если вы инвестор, то не имеет смысл вкладывать свои деньги
		до решения пункта 3. А лучше после п.4.
	\end{block}
\end{frame}

\section{Mouse Track Analysis}\label{section:mca}

\begin{frame}{Академические работы}
	\footnotesize
	\begin{enumerate}
		\item 
		диссертация
		С.М.Диденко
		\href{http://www.tmnlib.ru/jirbis/files/upload/abstract/05.13.18/605.pdf}{<<Разработка и исследование компьютерной модели динамики системы Пользователь-Мышь>>}
		\item 
		Praveen Sundar,
		A.V. Senthil Kumar
		\href{https://www.researchgate.net/publication/309558781_An_Improved_User_Verification_in_Online_Learning_Systems_Using_Behavioral_Profile}{<<An Improved User Verification in Online Learning Systems Using Behavioral Profile>>}
		\item Eric Hehman,
		Ryan Stolier,
		Jonathan B Freeman
		\href{https://www.researchgate.net/publication/276140699_Advanced_mouse-tracking_analytic_techniques_for_enhancing_psychological_science}{<<Advanced mouse-tracking analytic techniques for enhancing psychological science>>}
		\item 
		Bassam Sayed,
		Issa Traore,
		Isaac Woungang,
		Mohammad S. Obaidat
		\href{https://www.researchgate.net/publication/258792177_Biometric_Authentication_Using_Mouse_Gesture_Dynamics}{Biometric Authentication Using Mouse Gesture Dynamics}
		\item 
		Nazirah Abd. Hamid,
		Siti Dhalila Mohd Satar,
		Safie, S.,
		Suriayati Chuprat,
		\href{https://www.researchgate.net/publication/249315717_Mouse_Movement_Behavioral_Biometric_Systems}{<<Mouse Movement Behavioral Biometric Systems>>}
		\item ...
	\end{enumerate}
	\auditorium{Помним сказку о глупом мышонке...}
\end{frame}

\begin{frame}{Сырые данные MTA}
	\small
	В MTA (Mouse Track Analysis) существуют следующие сырые данные:
	\begin{itemize}
		\item $x, y$ -- координаты x, y курсора относительно окна
		\item $page\_x$, $page\_y$ -- координаты x, y курсора относительно страницы.
		На практике $page\_x \equiv x$, так как вбок не скролят. 
		\item $time\_stamp$ -- временная метка с определённой точностью.
		На практике лучше с точностью до 1/100 секунды. 
		Однако JS работает не стабильно и это вносит дополнительные проблемы.
		\item $which$ -- 0 если никакая клавиша не нажата (99.9\%),
		1 -- нажата основная клавиша (левая, если правша; правая если левша),
		2 -- вторичная,
		3 -- колесо мыши
		4,5 и т.д. -- клавиши игровых мышек.
	\end{itemize}
\end{frame}

\begin{frame}{Нормализация данных}
	\small
	С данными у которых произвольный $time\_stamp$ неудобно работать. 
	Разумно их привести к нормализованному виду с определённым шагом
	(например 10 или 25 мс.).
	
	В этом случае можно сохранить только $time\_stamp(0)$ -- временную метку начала трека мыши.

	Так же разумно удалить выбросы перед нормализацией.

	Методы нормализации:
	\begin{enumerate}
		\item \termdef{линейная интерполяция}.
		Это элементарный способ. Достаточно эффективен, если количество отсчётов в JS
		высоко. Однако неточности JS отсчётов не удаляются
		\item \termdef{(n, m, k) ансамбль полиномов} -- берем полиномы $n$-ой степени от $m$ точек. 
		Усредняем $k$ соседних полиномов в окрестностях точки $(x,y)$.
		\item \termdef{Фильтр Калмана} -- \auditorium{ДЗ}
	\end{enumerate}
\end{frame}

\begin{frame}{Два подхода к решению задачи курсорного почерка}
	После нормализации, можно решать задачу одним из двух способов:
	\begin{enumerate}
		\item Нейронные сети
		\item Классическое машинное обучение
	\end{enumerate}
\end{frame}

\begin{frame}{Нейронные сети}
	Варианты:
	\begin{enumerate}
		\item на каждого пользователя создаётся своя нейронная сеть: свой/чужой.
		Недостатки: мало обучающего материала для класса "свой";
		огромные вычислительные издержки если пользователей много 
		(ведь на каждого пользователя своя нейронная сеть)
		\item решение общей задачи фрод-нефрод.
		Недостатки: проблема переобучения; плохие априорные вероятности в 
		диапазонах от 0.2 до 0.8;
		мало данных для обучения (фродовых данных по прежнему МАЛО)
	\end{enumerate}

	Однако нейронные сети -- это будущее MTA, при условии если данных 
	трека мыши пользователя действительно много!
\end{frame}

\begin{frame}{Машинное обучение}
	Необходим feature extraction. Но на каком подмножестве данных его производить?
	
	Как показывает практика, сессии лучше разбивать на <<всплески>> -- 
	это кусочки движений мыши, ограниченные правилом:
	\begin{equation}
	time\_stamp_{i+1} - time\_stamp_{i} < \bigtriangleup_{t}
	\end{equation}
	где $ \bigtriangleup_{t}$ -- некая константа, показывающая 
	когда нужно разрывать <<всплески>>.
	
	\auditorium{Как определить величину $ \bigtriangleup_{t}$ для определённой выборки данных?}
\end{frame}

\begin{frame}{Feature Extraction}
	Разумно брать минимум, максимум, среднее, медиану, 5, 10, 25, 75, 90, 95 перцентили,
	среднее квадратичное и иные метрики над величинами:
	
	\begin{enumerate}
		\item скорость
		\item скорость по X
		\item скорость по Y
		\item ускорение
		\item jerk (ускорение ускорения)
		\item угловая скорость
		\item высота от точки кривой траектории до линии, соединяющую первую и последнюю точки
		\item ...
	\end{enumerate}
\end{frame}

\begin{frame}
	% Результаты на одном заказчике (2019)
	\includegraphics[width=11.5cm]{../pic/ueba_mta_example_reluslt.jpg}
\end{frame}

\begin{frame}{Лабораторная работа №2}
	Итак, вы готовы выполнить первую часть лабораторной работы №2.
	
	См. ссылку: 
	
\end{frame}


\section{Mobile Entity Analitics}\label{section:mea}
% TODO часть Ольги Шкряба

\begin{frame}
	К сожалению второй докладчик заболел.
	
	Приходите осенью 2020.
\end{frame}

% \section{Другие UEBA задачи анализа непрерывных потоков}\label{section:ueba_other}
% Eye Tracking
% Полиграф
% ...


\section{Вопросы для самопроверки}

\begin{frame}{Общие вопросы}
	\begin{enumerate}
		\item Как расшифровывается аббревиатура UABA? 
		\href{https://finance.rambler.ru/business/36508148-13-aprelya-v-moskve-id-kommersant-provedet-biznes-branch-user-behavior-analytics-uba-modnyy-trend-ili-novoe-effektivnoe-reshenie/}{Почему Комерсантъ предлагает аббревиатуру UBA?}
		\item Что такое raw data, stream data и track data? Приведите примеры. Приведите примеры, которые не были в лекции.
		\item Что такое refusal? Зачем в UEBA нужна эта величина?
		\item Расскажите <<cказку о глупом мышонке>>.
		\item Что такое Д.Н.К.?
	\end{enumerate}
\end{frame}

\begin{frame}{Какую задачу решаем?}
		\begin{enumerate}
		\item Что такое Д.Н.К.?
		\item Предложите Д.Н.К. для решения задачи третированной рекламы c помощью данных UEBA.
		\item Предложете Д.Н.К. для решения задачи обнаружения мошенничества среди пенсионеров c помощью данных UEBA.
		\item В чем различаются задачи: обнаружение мошенничества, детектирования свои/чужие, обнаружение аномалий в поведении? Как бы вы решали эти задачи? Что входило бы в класс 0, что в класс 1?
	\end{enumerate}
\end{frame}

\begin{frame}{Точность и полнота свои/чужие.}
\begin{enumerate}
	\item Аналогично формулам 
		\eqref{eq:ueba_self_presicion_def}
		и
		\eqref{eq:ueba_presicion_def}
		можете ввести понятия полноты на своих для пользователя $i$ и полноты для чужих
		для пользователя $i$?
	\item Есть ли смысл введения понятия полноты для своих для пользователя $i$ и полноты для чужих для пользователя $i$? 
	\item Аналогично \eqref{eq:ueba_self_precision_equation} 
		и \eqref{eq:ueba_presicion_def_equation}
		можете ввести понятия полноты определения своих и полноты [определения чужих]?
	\item Есть ли смысл введения понятия полноты определения своих и полноты [определения чужих]? 
\end{enumerate}
\end{frame}


\end{document}