\documentclass[russian,english,16pt]{article}
\usepackage[T1,T2A]{fontenc}
\usepackage[utf8]{inputenc}


\usepackage[russian, english]{babel}

\usepackage{hyperref}

%
%\usepackage[...]{hyperref}


\newcommand{\termdef}[1]{\textbf{\textit{#1}}}
\newcommand{\term}{\textit}

%\setcounter{secnumdepth}{4}


\title{Шпаргалка по курсу \href{https://github.com/kib-courses/dsis}{DSIS}}
\date{Осенний семестр\\ 2019}
\author{\href{mailto:pavelmstu@stego.su}{Слипенчук Павел Владимирович}}
\begin{document}
\maketitle

\section{Терминология}

Терминология представлена в алфавитном порядке. Англоязычные термины указаны после русскоязычных 

Обозначения:
\begin{enumerate}
	\item В \textit{<треугольных скобках>} указана лекция в которой рассказывается о данном понятии.
	\item В \textit{(круглых скобках)} указано стандартное математическое обозначение
	данного понятия.
\end{enumerate}



\paragraph{Контрибутор (контрибьютер)}
<02>
Это совокупность \term{признаков} (возможно один), который вносит определённый вклад в скоринговую модель.

\paragraph{Признак (feature)}\label{ru_feature}
($x_i$) <02>
Определенное значение. Категориальное, сравнимое, или числовое: целочисленное, булевое, или дробное

\paragraph{Feature} -- см. \hyperref[ru_feature]{признак}.

\section{Кванторы}

\section{Список литературы}
\begin{enumerate}
  \item ..
\end{enumerate}


% \nocite{SDMVol1}
% \nocite{SDMVol2}
% \nocite{SDMVol3}
% \nocite{SDMVol4}
% \nocite{Optim}
% \nocite{Fog}
% \nocite{InstSet}
% \nocite{Pentium}
% \nocite{Guk8086}
% \nocite{GukPen3}

% \bibliographystyle{unsrt}
% \bibliography{program}

\end{document}