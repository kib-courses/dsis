\documentclass{beamer}
\usetheme{metropolis} 
\usecolortheme{rose}

\usepackage{xcolor}
\usepackage[utf8]{inputenc}
\usepackage{hyphenat}
\usepackage[russian,english]{babel}          % Use metropolis theme

\setbeamertemplate{footline}[frame number] % указывает на каждой странице общее количество страниц




% Указывайте все новые термины в \termdef команде. А уже известные ранее или из других курсов в \term
\newcommand{\termdef}[1]{\textbf{\textit{#1}}}
\newcommand{\term}{\textit}
% Диалог с аудиторией.
\newcommand{\auditorium}[1]{\color{red}{\textbf{#1}}}
% \setbeamercolor{auditorium}{fg=red}

\title{Лекция 2. Экспертные системы. Основные параметры качества.}
% \date{\today}
\date{9 сентября 2019}
\author{Павел Владимирович Слипенчук}
\institute{Москва, МГТУ им.Бауманка,\\ каф.ИУ-8, КИБ}
% \titlegraphic{\includegraphics[width=2cm]{logo_ur.jpg}}
\titlegraphic{\small \href{https://github.com/kib-courses/dsis}{Data Science для решения задач информационной безопасности}}

\begin{document}
  \maketitle
    
  \begin{frame}{План лекции}
    \begin{enumerate}
	\item \nameref{section:classification_defs}
	\item второе
	\end{enumerate}
 \end{frame}
    
  \section{Признак. Вектор признаков Классы. Обучающая и тестовая выборки. Задача классификации, классификатор(оценщик)}\label{section:classification_defs}
  
  \begin{frame}{Признак, вектор, класс}
    \termdef{Признак} $x_i$-- определенное значение. Категориальное, сравнимое, или числовое: целочисленное, булевое, или дробное.
    
    \termdef{Вектор признаков} $\bold x = (x_1, x_2, ... x_n)$ -- вектор, каждое значение которого является \term{признаком}. 
  	
  	\termdef{Класс} (метка) $y$ -- значение (как правило целочисленное), присваиваемое какому-либо вектору признаков: $ y \mapsto \bold x$
  	
  	\auditorium{А в чем физический смысл?}
  \end{frame}
  
  \begin{frame}{Пример}
  \begin{itemize}
  	 \item $x_1$ -- сумма транзакции [в рублях]
  	 \item $x_2$ -- возраст клиента [в годах]
  	 \item $x_3$ -- пол клиента [булевый: 1 -- мужской, 0 -- женский]
  	 \item $x_4$ -- MCC код\footnote{\termdef{Merchant Category Code} -- номер деятельности компании при осуществлении безналичной оплаты. Например \textbf{1731} означает оплату за электроэнергию, \textbf{3137} -- покупка авиабилетов, \textbf{4121} -- такси}
  	 \item  $y=1$ -- операция мошенническая (фродовая);  $y=0$ -- легитимная.
  \end{itemize}
  \begin{center}\small \begin{tabular}{ l l }\label{tabular:class_feature_vector_example}
  	$0 \mapsto (3234, 25, 1, 1731) $ &  $0 \mapsto (2540, 55, 0, 1731)$ \\
  	$1 \mapsto (18400, 45, 0, 3137)$ & $0 \mapsto (2540, 55, 0, 1731)$  \\
  	$1 \mapsto (903, 19, 0, 4121)$  & $0 \mapsto (1875, 45, 0, 4121)$  \\
  	$0 \mapsto (854, 21, 1, 4121)$  & $1 \mapsto (702, 21, 0, 4121)$  \\
  	$1 \mapsto (903, 19, 0, 4121)$  & $0 \mapsto (1875, 45, 0, 4121)$  \\
  	$0 \mapsto (28400, 41, 1, 3137)$ & $0 \mapsto (25040, 55, 0, 1731)$  \\
  \end{tabular}\end{center}
  \end{frame}
  
  \begin{frame}
   \begin{block}{Замечание}
	  В отличие от таблицы, представленной на слайде №\ref{tabular:class_feature_vector_example},
	  в данных на реальных задачах \term{вектор признаков} может состоять из $~200$ и более признаков!
  \end{block}
  \end{frame}

  
  
  
  \begin{frame}{Классификация}
    \termdef{Классификация} -- это одна из задач \term{машинного обучения}, для которой каждому
    вектору признаков $\bold x = (x_1, x_2, ..., x_n)$ присваивается какой-либо \term{класс}.
    \newline
    \begin{center}
    	\includegraphics[width=8cm]{pic/classification_example.png}\centering
    \end{center}
	
  \end{frame}

   
   \section{Домашнее задание}

   \begin{frame}{Вопросы}
 	Посмотрите на \term{выборку} со слайда №\ref{tabular:class_feature_vector_example}. 
 	\begin{enumerate}
 	\item Верно ли утверждение, что если совершается оплата электроэнергии, то данная операция всегда легитимная?
 	\item Есть ли корреляция между возрастом клиента и мошенничеством при покупке авиабилетов? 
 	\item Можно ли предположить, что таксисты чаще обманывают молодых девушек? 
 	\item Покупает ли молодёжь авиабилеты через данный банк? 
 	\item Если совершают мошенничество в сфере автоперевозок, то обманывают мужчин или женщин?
	\end{enumerate}
	\end{frame}

\end{document}