\documentclass[english,russian,12pt]{article}
\usepackage[T1,T2A]{fontenc}
\usepackage[utf8]{inputenc}
\usepackage{babel}
% \usepackage{amsmath}
\usepackage{amsmath}
\usepackage{amsfonts}
\usepackage{amssymb}


\title{Вопросы и задачи к собеседованию по DSIS}
\date{10 сентября \\ 2019}
\author{Слипенчук Павел Владимирович}
\begin{document}
\maketitle


\subsection*{Вопрос №1}

Что такое математическое ожидание? Что такое медиана?

\subsection*{Вопрос №2}
Формулы сочетания, сочетания с повторением: 
$C_N^k$, $\overline{C_n^k}$. 

Применение на практике. 
Реальные задачи.

\subsection*{Вопрос №3}
Формулы размещения, размещения с повторением: 
$A_N^k$, $\overline{A_n^k}$.

Применение на практике. Реальные задачи

\subsection*{Вопрос №4}
Перестановки. 
Почему $P_n = A_n^n$?

Запись перестановок через циклы.
Объясните, почему любая перестановка может быть записана
единственным образом (с точностью до порядка)
в виде \textit{циклов}.
Например:
\begin{equation}
\left( 
	\begin{array}{ccccc}
	1 & 2 & 3 & 4 & 5 \\
	2 & 3 & 1 & 5 & 4
	\end{array}
\right) = (1, 2, 3) (4, 5)
\end{equation} 

\subsection*{Вопрос №5}
Квантиль и перцентиль.
25, 50, 75, 90, 95, 99 перцентили.
Применение 99 и 99.9 перцентиля (и других “высоких” перцентилей) для проблем высоко нагруженных систем.

\subsection*{Вопрос №6}
Что такое событие? 
Совместные и несовместные события.
Правила умножения и сложения.

\subsection*{Вопрос №7}
Условная вероятность. 
Формула Байеса. 
Вывод формулы Байеса.

\subsection*{Вопрос №8}
Граф. Дерево. Бинарное дерево.
Представление алгоритма в виде блок схемы.
Переход от блок схемы к бинарному дереву. 


\subsection*{Вопрос №9}
Совершенная конъюнктивная нормальная форма.
Совершенная дизъюнктивная нормальная форма.
Полином Жегалкина.

\subsection*{Вопрос №10}
Метод Монте-Карло.
Применение метода для решения инженерный задач: примеры.

\subsection*{Вопрос №11}
Определение вероятности статистическим способом
(<<апостериорная вероятность>>)
и аналитическим способом
(<<априорная вероятность>>).


\newpage
\subsection*{Задача №1}
Введём обозначения:
\begin{enumerate}
	\item $F_r$ (fraud real) -- событие, означающее что банковская операция в действительности является мошеннической. 
	\item $F_s$ (fraud system) -- событие, означающее, что некая система фрод-мониторинга оценила данную операцию как мошенническую.
	\item $L_s$ (legitim system) -- событие, означающее, что некая система фрод-мониторинга оценила данную операцию как легитимную (не мошенническая).
\end{enumerate}

Определим величину под названием <<полнота>> через условную вероятность:
\begin{equation}
\Pi \stackrel{def}{=} P(F_s | F_r)
\end{equation}

Докажите, что полноту можно высчитать по формуле:
\begin{equation}
\Pi = \frac{P(F_s \cap F_r)}{P(F_s \cap F_r) + P(L_s \cap F_r)}
\end{equation}


\subsection*{Задача №2}
Введём обозначения:
\begin{enumerate}
	\item $F_r$ (fraud real) -- событие, означающее что банковская операция в действительности является мошеннической. 
	\item $F_s$ (fraud system) -- событие, означающее, что некая система фрод-мониторинга оценила данную операцию как мошенническую.
	\item $L_r$ (legitim real) -- событие, означающее что банковская операция в действительности является легитимной (не мошенническая).. 
\end{enumerate}

Определим величину под названием <<точность>> через условную вероятность:
\begin{equation}
T \stackrel{def}{=} P(F_r | F_s)
\end{equation}

Докажите, что точность можно высчитать по формуле:
\begin{equation}
T = \frac{P(F_s \cap F_r)}{P(F_s \cap F_r) + P(L_r \cap F_r)}
\end{equation}


\subsection*{Задача №3}
Введём обозначения:
\begin{enumerate}
	\item $F_r$ (fraud real) -- событие, означающее что банковская операция в действительности является мошеннической. 
	\item $F_s$ (fraud system) -- событие, означающее, что некая система фрод-мониторинга оценила данную операцию как мошенническую.
	\item $L_r$ (legitim real) -- событие, означающее что банковская операция в действительности является легитимной (не мошенническая).
 	\item $L_s$ (legitim system) -- событие, означающее, что некая система фрод-мониторинга оценила данную операцию как легитимную (не мошенническая).. 
\end{enumerate}

Определим величину под названием <<точность>> через условную вероятность:
\begin{equation}
T \stackrel{def}{=} P(F_r | F_s)
\end{equation}

Определим величину под названием <<полнота>> через условную вероятность:
\begin{equation}
\Pi \stackrel{def}{=} P(F_s | F_r)
\end{equation}

Определим ошибки первого и второго рода.
\begin{equation}
O_1 \stackrel{def}{=} P(F_s | G_r)
\end{equation}
\begin{equation}
O_2 \stackrel{def}{=} P(G_s | F_r)
\end{equation}

Выведите $O_1$ и $O_2$ через $T$, $\Pi$, $P(F_r)$ и $P(L_r)$.

\newpage
\subsection*{Задача №4 (*)}
Миша и Вася поспорили, какова вероятность выпадения орла на конкретной монетке. 

Миша рассуждает так. \textit{<<Монетка всегда упадёт, она не может застрять в небе или исчезнуть. Вероятность что она упадёт на ребро будем считать настолько малой, что примем её за ноль. Таким образом она упадёт либо орлом, либо решкой. Однако ни одна из сторон ничем не лучше и не хуже другой. Из всего вышесказанного можно на языке математики составить простую систему уравнений:}
\begin{equation}
\left\{
\begin{array}{l}
p_{\text{орёл}} = p_{\text{решка}} \\
p_{\text{орёл}} + p_{\text{решка}} = 1
\end{array}
\right.
\end{equation}
Из этой системы уравнений легко понять, что $p_{\text{орёл}}=0.5$.

А Вася сказал следующее: \textit{Зачем философствовать? А ну дай монету сюда!} Затем Вася 100 раз подбросил её. Из 100 раз 54 раза выпал орёл. 

Вася сказал:  $p_{\text{орёл}}=\frac{54}{100}=0.54$.

Кто из мальчиков прав? 

Что должен сделать Вася, чтобы его метод оказался <<более правильным>>? 

Почему на практике метод Миши работает крайне редко? Можно ли на сложных задачах объединить методы Васи и Миши?


\end{document}