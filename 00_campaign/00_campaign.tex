\documentclass{beamer}
\usetheme{metropolis} 
\usecolortheme{rose}

\usepackage{xcolor}
\usepackage[utf8]{inputenc}
\usepackage{hyphenat}
\usepackage[russian,english]{babel}          % Use metropolis theme
\usepackage{wrapfig}

\setbeamertemplate{footline}[frame number] % указывает на каждой странице общее количество страниц

% Указывайте все новые термины в \termdef команде. А уже известные ранее или из других курсов в \term
\newcommand{\termdef}[1]{\textbf{\textit{#1}}}
\newcommand{\term}{\textit}
% Диалог с аудиторией.
\newcommand{\auditorium}[1]{\color{red}{\textbf{#1}}}
% \setbeamercolor{auditorium}{fg=red}

\title{Вводная лекция. Цели и задачи курса}
% \date{\today}
\date{3 сентября 2019 \\ 10 сентября 2019}
\author{Павел Владимирович Слипенчук}
\institute{Москва, МГТУ им.Бауманка,\\ каф.ИУ-8, КИБ}
% \titlegraphic{\includegraphics[width=2cm]{logo_ur.jpg}}
\titlegraphic{\small \href{https://github.com/kib-courses/dsis}{Data Science для решения задач информационной безопасности}}

\begin{document}
  \maketitle
    
   \begin{frame}{План лекции}
   \begin{enumerate}
   	\item \nameref{section:goals}
   	\item \nameref{section:topics}
   		\item \nameref{section:how_to}
   		\item \nameref{section:quastions}
   		\item \nameref{section:literature}
   	\end{enumerate}
   \end{frame}
  
  
  \section{Цели. Требования на входе. Результат.}\label{section:goals}
    
  \begin{frame}{Цели студента}
  	\begin{itemize}
  		\item Основы Data Science -- это часть
  <<джентльменского~минимума>>
  образованного профессионала 
  в сфере IT/ИБ 
  		\item Понимание границ задач ИБ,
  для которых методы машинного обучения
  применимы и возможны
  		\item Применение на практике знаний
  по математике
  		\item Опыт построения программных систем
  		\item Рекомендация на практику/трудоустройство 
  в сфере анализа данных / Data Science / Data Engineering
  		\item \textbf{Нормальные} темы для НИРС / дипломов
  	\end{itemize}
  \end{frame}
  
     
  \begin{frame}{Цели педагога}
  	\begin{itemize}
  		\item Рекрутинг
  		\item Развитие речи
  		\item Формирование <<ML \& DS культурной среды безопасников>>
  		\item Самообразование\footnote{
  		Всегда попадаются 2-3 умных студента, которые задают интересные и полезные для педагога и его саморазвития вопросы.}
  	    \item снять лапшу с ушей! :)
  		\item It is fun!
	\end{itemize}
  \end{frame}
  
  \begin{frame}{Требования на входе}
  \begin{enumerate}
  	\item Linux
  	\item Студент должен иметь уверенный навык программирования
  	(в предпочтении Python).
  	\item Знание Python или желание быстро и самостоятельно освоить этот 
  	язык программирования
  	\item Знание комбинаторики, теор.вера, мат.стата, дискретной математики
  	\item Алгоритмы и структуры данных
  \end{enumerate}
  \end{frame}

  \section{Темы спецкурса}\label{section:topics}
  
  \begin{frame}
 	%  \newcounter{topicscounter}
	\begin{enumerate}
	% \setcounter{topicscounter}{1}
	\item Экспертные системы, машинное обучение, Data Science, Data Engineering.
	\item Проблемы ИБ решаемые экспертными системами.
	\item Проблемы ИБ НЕ решаемые экспертными системами ;)
	\item Джентельменский минимум по ML: гиперплоскость, регрессия, функция штрафа, полнота и точность, VaR, SSI, ROC-кривая и AUC, понятие ансамбля, бутстрепинг, бустинг, бэггинг.
	\item Random Forest как наиболее простой и эффективный алгоритм. Его преимущества для 
	<<задач с противником>>.
	\item[6] Feature Extraction в ИБ задачах
	\end{enumerate}
\end{frame}

  \begin{frame}
\begin{enumerate}
	% \setcounter{topicscounter}{5}
	\item[7] UEBA: keyboard dynamics
	\item[8] UEBA: predilactions
	\item[9] UEBA: mouse track analysis
	\item[10] UEBA: алгоритм детектирования социальной инженерии
	\item[11] RSA\footnote{Имеется в виду система фрод-мониторинга RSA, а не асимметричный алгоритм шифрования :)} 
	\item[12] Brand Protection и Anti Piracy.
	\item[13] Бизнес-процессы в DS
	\item[14] Как НЕ работают нейронные сети :)
\end{enumerate}
\end{frame}

  \section{Как попасть на спецкурс?}\label{section:how_to}
  
  \begin{frame}{Собеседования}
  
  03 и 10 сентября, во вторник, в 10:15 в НОЦ ИБ, класс 6
  будет проведено собеседование. 
	
  По результатам будут отобраны ребята на спецкурс.

   \begin{block}{Замечание}
   	Собеседование -- это не экзамен. Вы такие, какие есть. 
   	Имеет смысл освежить в памяти ряд вопросов 
   	(см. <<\nameref{section:quastions}>>).
   	
   	Спецкурс читается каждый год. Не пройдёте сейчас -- 
   	пройдёте в следующем году
   \end{block}
  
  \end{frame}
  
  \begin{frame}{На кого рассчитан спецкурс}
  На студентов 3-4 курсов, уже <<набившие руку>> в программировании,
  имеющие базовые представления в алгоритмах и структур данных,
  знающие комбинаторику, теор.вер., матстат. на инженерном уровне.
  
  \end{frame}

  \begin{frame}{Студенты не из МГТУ им.Баумана}
  Если вы не учитесь в МГТУ им.Баумана, вам нужен пропуск.
  
  Для этого напишите в телеграмм \textbf{Александре Молоденовой}: @solinenarany
  
  Приходите на собеседование 10 сентября. 
  \textbf{Не опаздывайте}
  
  При успешно пройденном собеседовании вам выпишут его на семестр.
  \end{frame}
  
  \section{Вопросы к собеседованию}\label{section:quastions}
  
  \begin{frame}{Комбинаторика}
  \begin{enumerate}
  	\item Биномиальный коэффициент. Основные комбинаторные схемы: сочетание, размещение, перестановка, размещение с повторением, сочетание с повторением.
  	\item Формула включения/исключения
  	\item Подстановка
  	\item ~<<Урновые схемы>>
  	\item Отображения. Подсчёт количества отображений
  \end{enumerate}
  \end{frame}
  
  \begin{frame}{Теория вероятности}
  \begin{enumerate}
    \item Вероятность как статистическая величина и как априорная величина
    \item Закон сложения и закон умножения
    \item Условные вероятности. Формула Байеса
    \item Ошибка первого и второго рода
    \item Связь теории вероятности и комбинаторики
   \end{enumerate}
   \end{frame}
 
  \begin{frame}{Математическая статистика}
  \begin{enumerate}
    \item Математическое ожидание, медиана, мода
    \item Момент случайной величины
    \item Квантиль, перцентиль
    \item Нормальное распределение
    \item Корреляция
    \item ЦПТ
    \item Доверительный интервал
    \item Метод Монте-Карло
    \item Связь математической статистики и теории вероятности
 \end{enumerate}  
\end{frame}

  \begin{frame}{Алгоритмы и структуры данных}
   \begin{enumerate}
   	
     \item массив, очередь, список
     \item хеш-таблица, словарь
     \item дерево.
     \item дерево поиска. Примеры.
     \item алгоритм Дейекстры, алгоритм Краскала
     \item P и NP задачи. NP-complete задачи. Проблема $ P\stackrel{?}{=} NP$.
     \item Задача коммивояжёра. Алгоритм Литтла.  
     \item Решение задачи математическое и решение для нужд бизнеса.   
	\end{enumerate}
  
	\end{frame}
  
  
  
  \section{Рекомендуемая литература}\label{section:literature}
  
  \begin{frame}{Python}
  
  Книги
  \begin{itemize}
  	\item Для начинающих: 
  	\textbf{Марк Саммерфилд. Python на практике} 
  	\item 
  	Для продвинутых:
  	\textbf{Лучано Рамальо. Python: к вершинам мастерства.}
  \end{itemize}
  
  Пишите много кода и придумывайте интересные задачи. 
  Например по прикладной стеганографии:
  \begin{itemize}
  	\item Русскоязычный чат-бот Boltoon: создаем виртуального собеседника
  	\url{https://habr.com/ru/post/340190/}
  	\item Хэш-стеганография с использованием vkapi
  	\url{https://habr.com/ru/post/351370/}
  \end{itemize}
\end{frame}
	
	\begin{frame}{Комбинаторика}
	
	Методичка Жуковых. 
	
	В канале по хештегу библиотека есть.
	
	\end{frame}

\begin{frame}{Теория вероятности и математическая статистика.}
Курс будет \textit{прикладным}, поэтому для его понимания 
нужно хорошо знать основные определения и их смысл.

Есть много хороших книг, например 
<<Феллер В. Введение в теорию вероятностей и ее приложения>>.

Если вы захотите стать профессиональным Data Scientist-ом,
то, конечно теор.вер и мат.стат нужно будет выучить <<на зубок>>.

Однако для подготовки к спецкурсу -- это излишне.

Смотри вопросы из секции <<\nameref{section:quastions}>>.
\end{frame}

   
  
\end{document}